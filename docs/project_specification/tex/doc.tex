\documentclass[12pt, oneside, titlepage, final]{article}

\usepackage[margin=1.0in,headheight=15pt]{geometry}
\usepackage{graphicx}
\usepackage{parskip}
\usepackage[pdfpagelabels]{hyperref}
\usepackage{microtype}

\graphicspath{{img}}

\hypersetup{colorlinks = true, urlcolor = cyan, linkcolor = black, citecolor = red}

\begin{document}

	\hypersetup{pageanchor=false}

	\begin{titlepage}
		\centering

		\vspace*{3cm}

		\Huge
		\textbf{Project Specification for Millimeter-wave People Detection Radar}

		\vspace{0.5cm}

		\large
		Wireless Embedded Systems Capstone\\
		by Steven Daniels and Matthew Hatch

		\today

		\vspace{3cm}

		\includegraphics[width=12cm]{ucsd_logo}

	\end{titlepage}

	\pagenumbering{Alph}

	\tableofcontents

	\newpage

	\hypersetup{pageanchor=true}

	\pagenumbering{arabic}

	\section{Project Charter}

	\subsection{Overview}

	This project aims to implement a system that uses a high-frequency, short-range radar to detect humans
	present in front of the radar. This detection will be in the form of a two-dimensional plot that shows
	the distance to the target and the target's motion relative to the radar, and the strength of the
	detection. This plot is called a “range-Doppler map". Additionally, if time permits, it also aims to
	produce a three-dimensional point cloud that allows visualization of the human targets in 3D space.

	\subsection{Approach}

	We will implement this system using a pre-built radar system called an Antenna-on-Platform (AOP)
	offered by Texas Instruments. The IWR6843AOPEVM combines a 60 GHz radar platform with a highly capable
	digital signal processing subsystem in a single system-on-chip. The AOP provides the analog frontend
	necessary to transmit and receive a Frequency-modulated Continuous Wave (FMCW) radar waveform; it
	handles generation of the frequency ramp, duty cycle management, mixing the transmit and receive
	signals, and I/Q sampling of the intermediate frequency signal.

	Since the AOP handles all aspects of waveform generation and reception, the project will focus on the
	digital baseband signal processing for the FMCW waveform. FMCW operates by transmitting a sequence of
	“chirps", which are short transmissions consisting of a carrier wave that linearly ramps in frequency
	over the duration of the transmission. The transmitted signal then enters the channel and reflects off
	of targets. The reflected signal then arrives at the receiving antenna; this signal is then mixed with
	the signal being transmitted to produce an “intermediate frequency" (IF) signal. The difference in
	frequency between the received signal and transmitted signal manifests as a tone in the IF signal; the
	frequency of the tone is used to estimate range to the target. If there are multiple targets
	reflecting energy at different ranges, then multiple tones will appear in the IF signal.

	Upon the reception of each subsequent chirp, a Discrete Fourier Transform (typically by Fast Fourier
	Transform) is performed to identify the tones present therein; this step of the baseband processing is
	called the Range or “fast-time" FFT.

	In order to estimate the velocity of the target, the radar transmits a fixed number of chirps; this
	collection of chirps is referred to as a “frame". As the radar transmits and receives chirps, it
	performs the fast-time FFTs and buffers the received signals. Once a full frame of chirps has been
	transmitted and received, the baseband processor performs the next step, which is a two-dimensional
	Discrete Fourier Transform over all of the IF signals of the chirps in the frame. This is referred to
	as the Doppler or “slow-time" FFT.

	The slow-time FFT observes the minute changes in the phase of the received signal over the course of
	the frame and uses them to estimate the Doppler effect as a result of the relative motion between the
	radar platform and the target. While the fast-time FFT utilizes the difference in frequency between
	the transmitted and received signal to estimate range, the slow-time FFT relies on changes in phase to
	estimate frequency.

	Once the Doppler frequency of the target is known, it is a simple algebraic problem to convert Doppler
	frequency to relative velocity.

	The IWR6843AOPEVM features a baseband digital signal processor (the TMS320C674x) and an onboard
	hardware accelerator that includes cores for common FMCW processing blocks, including Fast Fourier
	Transforms. Additionally, Texas Instruments provides a software SDK that includes libraries for
	interfacing with the board hardware and communicating between the analog frontend and digital signal
	processing subsystem. Therefore, this project will primarily consist of developing an implementation
	of the range and Doppler processing steps and then utilizing the SDK to implement and verify the
	processing in hardware. We will then use Matlab or a similar tool to plot and visualize the output of
	the processing in real time to show that the radar can successfully image a human target.

	\subsection{Minimum Viable Product}

	The minimum viable product for this project is a demonstration that the IWR6843AOPEVM can be attached
	to a Windows PC and, using the software we have written, can successfully produce a range-Doppler map
	and display a plot of it in real time on the host PC\@. This range-Doppler map should show clean,
	strong returns for targets in front of the radar. We intend to demonstrate the range-Doppler map
	generation using corner reflectors (stationary and moving) and human targets. We also want to
	demonstrate that it can show returns for multiple targets at different ranges and velocities in its
	field of view.

	Once the minimum viable product is achieved, the next step would be to implement radar tracking and
	clutter-removal algorithms. These algorithms process the range-Doppler data and perform two important
	functions that would allow the radar to be used in a real industrial application; they eliminate false
	returns from static objects in the environment that obscure the detection of actual targets of
	interest (i.e.\ humans), and then summarize the range-doppler map into a collection of identified
	targets by applying a detection threshold. The canonical algorithm for this purpose is called Constant
	False Alarm Rate (CFAR). The IWR6843AOPEVM hardware accelerator has built-in support for the
	processing involved in the CFAR algorithm.

	Following that, we plan to leverage the fact that the IWR6843 radar is a MIMO system; it features 4
	receive and 3 transmit antennas. The algorithms thus far have depended on and only use a single
	baseband I/Q stream from one of the receive antennas. However, the use of only a single transmit and
	receive antenna means that the radar cannot resolve targets in 3D space; it is only capable of
	estimating range and relative motion. The addition of multiple receive antennas allows observing the
	difference in phases between the incident received signals, which in turn can be used to estimate the
	angle of arrival (AoA, or azimuth) of the signal (in a plane around the radar with two antennas, or in
	three-dimensional space with three antennas). Since the IWR6843 possesses three receive antennas, it
	is possible to estimate the position of an object in 3D space relative to the radar by combining
	angle-of-arrival estimation with range estimation.

	By combining CFAR target detection with AoA processing, the long-term goal for the project is to
	produce a three-dimensional point cloud visualization of the radar's field of view.

	\subsection{Constraints, Risk, and Feasibility}

	Given the progress we have already made, the main roadblock to achieving the MVP will be understanding
	and successfully utilizing Texas Instruments' Millimeter Wave Radar SDK to create a radar processing
	pipeline. We have already obtained all of the required hardware, written and tested Matlab simulations
	of the range and Doppler processing, and performed checkouts to ensure that the radar platform is
	fully functional. The software SDK, however, is extremely powerful and fully-featured, but this means
	that it is also rather complicated and there is a vast amount of documentation to comb through to
	understand it. We have given ourselves plenty of time by completing all the other steps early, though,
	and so we anticipate we should be able to write and debug a custom radar processing pipeline within
	the quarter.

	Another potential pitfall is issues with the Texas Instruments evaluation boards; while we have
	already ordered, received, and checked out all the required hardware, there have been some issues with
	them working consistently; however, we have two copies of each of the boards, and the DCA1000EVM used
	for raw ADC capture is not strictly required for the project, so we do not consider this risk to be
	particularly worrisome.

	\section{Group Management}

	Thus far, we have mostly divided up work along a theory/implementation boundary; since Steven has more
	experience with radar systems and the theory behind them, and Matthew has experience primarily in
	communications systems and modems design, Steven has been the primary developer for the Matlab
	simulations, while Matthew has worked more on the hardware integration, including board checkout and
	testing and creation of Simulink models.

	We communicate via Discord, and since there are only two of us, haven't needed to establish a formal
	decision-making process. We have been managing schedule by following a sprint-based model of
	establishing a set of obtainable goals for two-week periods and aiming to complete at least those
	goals in the time frame. Thus it is easy to see when the schedule slips, and we are able to rearrange
	the estimated work for later sprints to account for any such issues. We have given ourselves some
	extra time in the planned schedule at the end of the quarter to account for any unanticipated schedule
	changes.

	\section{Project Development}

	The project will use an IWR6843AOPEVM radar evaluation platform, MMWAVEICBOOST carrier card which
	provides an integrated JTAG probe/debugger for loading code onto the AOPEVM's ARM processor and
	digital signal processor and then debugging the code running on the board. These two boards are
	minimum requirements for the project, as both are needed to develop for the AOPEVM\@. Additionally, we
	are using a DCA1000EVM capture card which allows real-time streaming of the raw analog-to-digital
	converter output of the IWR6843 analog frontend to a host PC for capture and analysis to aid in
	debugging and algorithm design. We have ordered two sets of this hardware and have already received
	and verified all the units work.

	We have a corner reflector that can be used as an easy-to-detect baseline target for the radar;
	combined with the fact that Texas Instruments supplies a reference implementation of the FMCW radar
	that generates a range-Doppler map, we are able to test our system by comparing the output of our
	custom-written processing pipeline against Texas Instruments'. We can also use the corner reflector in
	a controlled environment to determine what the `ground truth' is that we should be seeing in the
	system output.

	We have documented the algorithms development via heavily-annotated Matlab scripts and READMEs with
	plots of the output of the processing steps; we also plan to create Matlab live scripts which
	demonstrate the processing and produce visualizations of the algorithms inline with the code.
	Documentation for the embedded C++ code will be done by cross-referencing the C++ implementation
	against the corresponding Matlab implementation in a standard C++ documentation standard such as
	Doxygen-style doc comments.

	\section{Project Milestones and Schedule}

	The high-level concrete deliverables for our project are:

	\begin{itemize}
		\item Millimeter Wave Radar SDK application suite (DSP Subsystem and Mission Subsystem
			firmware) that outputs a range-Doppler map of the radar's field of view in real time,
			which can demonstrate the detection of human targets.
		\item Windows host PC software (e.g. Matlab script) that can start and stop the radar
			application without the need for Texas Instruments' Millimeter Wave Studio and
			visualize the output of the radar application.
		\item (Optional, time permitting) DSS/MSS firmware and visualizer for 3D point cloud of
			targets detected by the radar
	\end{itemize}

	We have organized these into biweekly sprints as follows:
	\begin{enumerate}
		\item April 5th to April 19th:
			\begin{enumerate}
				\item Using raw I/Q data captured from the IWR6843AOP ADC, validate that the
					Matlab scripts we have written produce results that are consistent
					with the range-Doppler map produced by Texas Instruments' own
					postprocessing. This will be handled by Steven.
				\item Redo the preceding item with a more realistic scenario; i.e.\ instead of
					pointing the radar at a small, empty room, put real targets in front
					of it (a corner reflector and a person) and verify that the results
					are consistent with the real target picture. We should be able to
					observe returns in the range-Doppler map for the target at the correct
					range and velocity. This will be handled by Steven.
				\item Perform an initial checkout of the MMWAVEICBOOST carrier card. This
					includes that it powers on, the USB connection to Windows is
					functional, and that the integrated JTAG debugger is usable through
					Code Composer Studio. Additionally, survey the documentation for the
					carrier card to determine the steps necessary to compile and flash MSS
					and DSS firmware on the IWR6843AOPEVM using the JTAG debugger. This
					will be handled by Matthew.
			\end{enumerate}
		\item April 20th to May 3rd:
			\begin{enumerate}
				\item Attempt to intercept the raw ADC data output by the DCA1000EVM over
					Ethernet and see if it can be streamed in real-time instead of merely
					being saved to a capture file. If so, consume the stream in Matlab or
					Simulink and run the range-Doppler processing on it to produce a
					real-time plot. This will be handled by Steven.
				\item Create a custom radar processing pipeline using the Millimeter Wave SDK
					that runs on the IWR6843AOPEVM; verify that the output can be read
					from the UART over USB on the host Windows PC, e.g.\ in Matlab. The
					output may initially just be a dummy pattern; the goal here is to gain
					an understanding of how to write and flash an application on the radar
					board. This will be handled by Matthew.
				\item Write a program (in Matlab or possibly a traditional programming
					language) to send commands to the Millimeter Wave application's
					command line interface over the UART, in order to set up the radar
					parameters and start and stop the radar. This will be handled by
					Matthew.
			\end{enumerate}
		\item May 4th to May 17th:
			\begin{enumerate}
				\item Convert the dummy custom pipeline to one that produces the desired
					range-Doppler map by implementing the range and Doppler FFTs in the
					pipeline, and any other requisite filtering necessary to clean up the
					signal. This will be handled by Matthew.
				\item Write a Matlab or Simulink script that reads the range-Doppler data off
					of the UART and plots it for viewing. This will be handled by Steven.
			\end{enumerate}
		\item The remaining time is allocated as slip time to account for unexpected delays in the
			planned schedule; if we finish the range-Doppler map deliverables with time to spare,
			the remaining time will go to attempting to build a 3D point cloud visualization.
	\end{enumerate}

\end{document}
